\documentclass{article}
\usepackage{amsmath}
\usepackage{graphicx}
\usepackage[margin=2cm, bindingoffset=1cm, inner=2cm]{geometry}
\usepackage{hyperref}
\usepackage{textcomp}
\usepackage{float}
\usepackage{amsmath}
\usepackage{listings}
\usepackage{color}

\definecolor{dkgreen}{rgb}{0,0.6,0}
\definecolor{gray}{rgb}{0.5,0.5,0.5}
\definecolor{mauve}{rgb}{0.58,0,0.82}

\lstset{frame=tb,
   language=Java,
   aboveskip=3mm,
   belowskip=3mm,
   showstringspaces=false,
   columns=flexible,
   basicstyle={\small\ttfamily},
   numbers=none,
   numberstyle=\tiny\color{gray},
   keywordstyle=\color{blue},
   commentstyle=\color{dkgreen},
   stringstyle=\color{mauve},
   breaklines=true,
   breakatwhitespace=true
   tabsize=3
   }

\title{\bf SIMPLE EXAMPLE: HEAT CONDUCTION}
\author{Sakina Rehman}

\begin{document}
\maketitle

\noindent This example focuses on solving a heat conduction problem, which is written in the weak formulation of partial differential equations (PDEs). Find the temperature $u \epsilon H^1(\Omega)$ such that for all $v \epsilon H_0^1(\Omega$) holds \newline


\[
    \int_\Omega v(\frac{\delta u}{\delta t}) + \int_\Omega c\nabla v.\nabla u = 0, \forall v, u(x,0) = g(x), u(x,t) = 
\begin{cases}
    -2   x \epsilon \Gamma_{left}\\
    -2   x \epsilon \Gamma_{right}
\end{cases}
\]

\noindent where c is the thermal diffusivity. The following steps are required to solve this problem using \emph{SfePy}:

\begin{itemize}

\item The domain $\Omega$ must be discretized to create a finite element mesh. The mesh can be loaded from the \emph{meshes} folder or alternatively generated by the code (simple shapes)

\lstset{language=Python}
\begin{lstlisting}
 filename_mesh = 'meshes/3d/cylinder.mesh'
\end{lstlisting}

\item Regions are domains of integration and allow the user to define the initial and boundary conditions. The code below defines the domain $\Omega$ and the boundaries $\Gamma_{left}$ and $\Gamma_{right}$

\begin{lstlisting}
regions = {
	'Omega' : 'all'
	'Left' : ('vertices in (x < 0.00001)', 'facet'),
	'Right' : ('vertices in (x > 0.099999)', 'facet'),
}
\end{lstlisting}

\item The field is defined as the discrete function spaces which can be defined using the number of components, region name, data type etc. The field can either be defined on a whole cell subdomain or on a surface region.

\begin{lstlisting}
fields = {
	'temperature' : ('real', 1, 'Omega', 1),
}
\end{lstlisting}

\item These discrete function spaces (FE spaces) can now be used to define variables. Variables can be in three forms: unknown field (for state variables), test (virtual) field and the parameter field, which is for variables with a known degree of freedom (DOF). The '1' in the code below shows a history size of 1, as the previous time step state is required for the numerical derivative. The value 'u' below is the name of the unknown variable.

\begin{lstlisting}
variables = {
	'u' : ('unknown field', 'temperature', 0, 1),
	'v' : ('test field',    'temperature', 'u'),
}
\end{lstlisting}

\item Materials can be given as the constant parameter c, as part of the material 'm'.

\begin{lstlisting}
materials = {
	'm' : ({'c' : 1.0e-5},),
}
\end{lstlisting}

\item The essential boundary conditions will also be set as constants. In this case, the value of u will be -2 and 2 on $\Gamma_{right}$ and $\Gamma_{left}$ respectively

\begin{lstlisting}
ebcs = {
	'u1' : ('Left', {'u.0' : 2.0}),
	'u2' : ('Right', {'u.0' : -2.0}),
}
\end{lstlisting}

\item To define the initial conditions, \emph{NumPy} must be imported. The initial conditions apply to the entire domain $\Omega$ and ic max is a constant defined outside the function

\begin{lstlisting}
import numpy as np

def get_ic(coors, ic):
	x, y, z = coors.T
	return 2 - 40.0 * x + ic_max * np.sin(4 * np.pi * x / 0.1)
functions = {
	'get_ic' : (get_ic,),
}
ics = {
	'ic' : ('Omega', {'u.0' : 'get_ic'}),
}
\end{lstlisting}

\item The PDEs are now built as a combination of linear predefined terms. Each term has its own quadrature order and a region of integration. The integral specifies a numerical quadrature order.

\begin{lstlisting}
  integrals = {
      'i' : 2,
  }
  equations = {
	'Temperature' : """dw_volume_dot.i.Omega(v, du/dt)
				+ dw_laplace.i.Omega(m.c, v, u) = 0"""
}
\end{lstlisting}

This simulation (full code in Appendix X) is then run in the Jupyter Notebook terminal using:

\begin{lstlisting}

sfepy-run simple poisson_short_syntax.py
\end{lstlisting}

\end{itemize}

\bibliographystyle{unsrt}
\bibliography{references}


\end{document}
